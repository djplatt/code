%% Based on a TeXnicCenter-Template by Tino Weinkauf.
%%%%%%%%%%%%%%%%%%%%%%%%%%%%%%%%%%%%%%%%%%%%%%%%%%%%%%%%%%%%%

%%%%%%%%%%%%%%%%%%%%%%%%%%%%%%%%%%%%%%%%%%%%%%%%%%%%%%%%%%%%%
%% OPTIONS
%%%%%%%%%%%%%%%%%%%%%%%%%%%%%%%%%%%%%%%%%%%%%%%%%%%%%%%%%%%%%
%%
%% ATTENTION: You need a main file to use this one here.
%%            Use the command "\input{filename}" in your
%%            main file to include this file.
%%

\documentclass[12pt,a4paper]{article}
\usepackage{url,amsfonts,fancyhdr}


\begin{document}
\pagestyle{fancy}
\setlength{\headheight}{15pt}
\setlength{\parindent}{0in}
\setlength{\parskip}{0.15in}
\fancyhead[L]{DJP}
\fancyhead[R]{10/Aug/2010}
\fancyhead[C]{Hurwitz Calculator Notes}
\newcommand{\N}{{\sf N\hspace*{-1.0ex}
\rule{0.15ex}{1.3ex}\hspace*{1.0ex}}}

Bian

Here are some notes on how to use the Hurwitz function calculator. The basic steps are:-
\begin{itemize}
\item Determine an exactly representable number close to the desired eigenvalue(s)
\item Edit z\_file.gp to call the PARI function zeta\_file once for each eigenvalue.
\item gp z\_file.gp
\item ed zetas1.txt $<$ ed.txt (repeat for each eigenvalue file produced)
\item ./l-func-mpfi-1.2 200 100 100 3150 0 16384 zetas1.txt bian1.dat 31 3150 30 (repeat for each eigenvalue file produced)
\end{itemize}
which results in the data file bian1.dat (etc. for each eigenvalue).

\subsection{Representable numbers}

It is important that the eigenvalues specified in the call to zeta\_file are exactly representable in IEEE double precision. This is what was causing us the problems earlier on with loss of accuracy. Thus $30.66365967$ is no good so I use $30.6636596700000012560849427245557308197021484375$ instead. A simple C++ program called double-rep will read a double precision number from the command line and print out an exactly representable one instead which should not be too far away. e.g.
\begin{verbatim}
dave@dave-desktop:$ g++ double-rep.cpp -odouble-rep
dave@dave-desktop:$ ./double-rep 18.90241257
Argument is 1.8902412569999999192305040196515619754791259765625000000000e+01
dave@dave-desktop:$ 
\end{verbatim}

\subsection{The PARI code}

We use Pari to create a file of Gamma and Zeta values (my code used the Gamma values but yours won't). The Pari source file z\_file.gp defines a function zeta\_file and calls it twice (once for each of two eigenvalues). The arguments to zeta\_file are:-
\begin{itemize}
\item N: The number of zeta values per s to print (use 101)
\item num\_s: The number of s to process (always 1 for you)
\item im\_s: The imaginary part of s (your eigenvalue). Must be exactly representable (see above)
\item step: The step between different s values (not used so 0.0 is ok)
\item fname: The file name to create e.g. zetas1.txt
\end {itemize}

The output file consists of the number of zeta values (101) followed by the eigenvalue followed by intervals representing
\begin{itemize}
\item $\Re\Gamma((1.0+i*im\_s)/2)$ (Not relevant to you)
\item $\Im\Gamma((1.0+i*im\_s)/2)$ (Not relevant to you)
\item $\Re\Gamma((2.0+i*im\_s)/2)$ (Not relevant to you)
\item $\Im\Gamma((2.0+i*im\_s)/2)$ (Not relevant to you)
\item $\Re\zeta(1.0+n+i*im\_s)$ for $n=0\ldots N-1$
\item $\Im\zeta(1.0+n+i*im\_s)$ for $n=0\ldots N-1$
\end{itemize}

\subsection{The ed step}

Unfortunately Pari and MPFI disagree slightly on format rules and we need to replace every occurance of a space followed by capital E with just capital E. I use the Unix ed command with the file ed.txt. e.g.
\begin{verbatim}
dave@dave-desktop:$ ed zetas2.dat < ed.txt 
90591
90178
dave@dave-desktop:$
\end{verbatim}

\subsection{l-func step}

The program l-func-mpfi-1.2.c is compiled with
\begin{verbatim}
dave@dave-desktop:$ gcc -O1 -static -march=core2 -ol-func-mpfi-1.2\
 l-func-mpfi-1.2.c -lmpfi -lmpfr -lgmp -lm
dave@dave-desktop:$
\end{verbatim}
You might get some warnings about redefining FAILURE and unused return values from fscanf but you can safely ignore those. The executable takes the following parameters.
\begin{itemize}
\item prec: number of bits of precision to use (use 200)
\item N: number of terms to use in taylor approximation (max 100) (use 100)
\item M: number of terms to calculate via Taylor (as opposed to simply summing the series) (use 100, i.e. calculate ALL terms by Taylor)
\item RN: internally work with $\zeta(s,\alpha)=\sum\limits_{n=RN}^{\infty}(n+\alpha)^{-s}$ (we use 3150 but can vary it if you like)
\item N\_SUM: how many terms to use when summing. Irrelevant because we do ALL terms via Tayor. (use 0)
\item NO\_GAPS: $\alpha$ steps from $1.0$ to $0.0$ in steps of $\frac{1}{NO\_GAPS}$ (use 16384)
\item data file: file from previous step (zetas1.dat)
\item out file: output file (e.g. bian1.dat)
\item N\_OUT: how many hurwitz values per alpha to output (31)
\item RN\_OUT: the values stored are $\zeta(s,\alpha)=\sum\limits_{n=RN\_OUT}^{\infty}(n+\alpha)^{-s}$. We use the same value for RN\_OUT as for RN (3150).
\item NUM\_DIG: how many decimal digits to output (30)
\end{itemize}

For example (output slightly reformatted)
\begin{verbatim}
dave@dave-desktop:$ time ./l-func-mpfi-1.2 200 100 100 3150 0 16384\
> zetas1.dat bian1.dat 31 3150 30

Running with prec=200 N=100 M=100 RN=3150 N_SUM=0 NO_GAPS=16384.
N_OUT=31 RN_OUT=3150.
Processing Im(s)=3.066365967000000126e+01
H_1(s,0.25)=
[-3.0247275475413589839223718941489212605356516557347945742748573e-2,
-3.0247275475413589839223718941489212605356516557347945742509511e-2] +
[1.2192059562072135024161146760075489356490474932310730057821686e-2,
1.2192059562072135024161146760075489356490474932310730057941626e-2]i
H_1(s,0.5)=
[-3.0217513845235411259982008217054903254861613049217476815669188e-2,
-3.0217513845235411259982008217054903254861613049217476815669168e-2] +
[1.2265636784546623482992656864886498840154951758264549431847219e-2,
1.2265636784546623482992656864886498840154951758264549431847230e-2]i
H_1(s,0.75)=
[-3.0187575620252984903781476602052188051155772599992038610187314e-2,
-3.0187575620252984903781476602052188051155772599992038609948174e-2] +
[1.2339135527093785718217485685286336022541883404972147629366870e-2,
1.2339135527093785718217485685286336022541883404972147629486713e-2]i
 56 55 56 55 55 56 55 56 55 55 56 55 55 56 55 55 56 55 56 55 55 56 55 
55 56 55 55 56 55 55 56.
Completed.

real	4m40.664s
user	4m39.920s
sys	0m0.680s
dave@dave-desktop:$ 
\end{verbatim}
The program prints the values of $\zeta_1(s,1/4)$, $\zeta_1(s,1/2)$ and $\zeta_1(s,3/4)$ and the relative accuracy in decimal places of the output at $\alpha=1/4$ which is where I'd expect the least accuracy. Here we are getting $>50$ places and only outputting $30$ of them (NUM\_DIG) to the data file so all is well.

The output file consists of decimal intervals for $\zeta(s+n,\alpha)$, real part then imaginary part, where $n$ goes from $0\ldots N\_OUT-1$ and $\alpha$ goes from $1.0$ down to $0.0$ in steps of $\frac{1}{NO\_GAPS}$ so the first three entries are $\Re\zeta_1(s,1.0)$, $\Im\zeta_1(s,1.0)$ and $\Re\zeta_1(s+1,1.0)$ with $\zeta_1(s,\alpha):=\sum\limits_{n=RN\_OUT}^\infty(n+\alpha)^{-s}$.

\end{document}